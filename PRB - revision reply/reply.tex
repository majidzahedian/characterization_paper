\documentclass[12pt]{amsart}
\usepackage{geometry} % see geometry.pdf on how to lay out the page. There's lots.
\usepackage[dvipsnames]{xcolor}
\geometry{a4paper} % or letter or a5paper or ... etc
% \geometry{landscape} % rotated page geometry

% See the ``Article customise'' template for come common customisations
% delete this line to display the current date

%%% BEGIN DOCUMENT
\begin{document}
	\color{red} \section{ Report of the First Referee}
	The authors compare nuclear spin characterization using ESEEM and DD
	sequences. Overall the work seems timely and appropriate for
	publication in PRB.
	
	However, I think some sections and figures would benefit from further
	revisions. Below is a list of comments and inquiries:
	
	1) In some places the authors refer to second order hyperfine
	coupling, I assume this is in regard to the Zeeman splitting of the
	central spin but this could be clarified in the text.
	
	2) I am not sure whether this is intentional but the equations miss
	punctuation and equation references the brackets.
	
	3) Some equations are hard to read because they overlap with the text
	of the next column.
	
	4) Related to 1) additional details to derive the Hamiltonian in Eq.
	(1) would be helpful, for instance when does this equation hold, I
	believe $\Delta \gg$ hyperfine elements is assumed to neglect coupling to Sx and
	Sy and then the nuclear spins are rotated such that there occur no
	terms with Iy? Is there a difference in the derivation for S=1 and
	S=1/2? Maybe an appendix applying the secular approximation and the
	appropriate rotation to the frame leading to this form would be
	helpful for future readers.
	
	5) Fig. 2 misses the labels for the subfigures the same applies to
	Fig. 3 where I think that additionally the heading of the top left
	panel should read "Electron Spin 1".
	
	6) In Eq. (2) it would help to provide the state that is used for the
	calculation as well as an intuition why it is sufficient to include
	the exponential decay term after solving the dynamics (and when this
	is valid).
	
	7) The last paragraph of App. B is very hard to read and would benefit
	from some work on the language.
	
	8) Could the authors comment on whether they have an intuition whether
	adapted approaches would be feasible for systems with an additional
	central spin like the (15N)V center or defects with even larger spin
	like vanadium defects in silicon carbide?
	
	9) Could the authors provide a short comment comparing to entanglement
	based approaches for nuclear spin control, e.g. : \\
	https://www.nature.com/articles/ncomms14660?
	
	Concluding, if the above points are addressed, I deem the article
	suited for publication within PRB.
	
	\color{black} \section*{Reply to the First Referee}
	We appreciate the comments made by the referee. We clarify each point as follows:\\
	
	1. We thank the Referee for the usefull comment, indeed we used a confusing  formulation. Second order hyperfine couplings mean terms like $(\frac{A_{zz}}{\omega_L})^2$ and $(\frac{A_{zx}}{\omega_L})^2$ in the resonance conditions. This is stated in the sentence:
	((... the second-order term $\frac{A_{zx}^2}{\omega_L^2}$ can be neglected ... ))
	We replaced this terms by "second order correction of the resonance with respect to hyperfine parameters", which in our opinion better reflects the meaning. 
	To clarify this point, we add the following explanation to the text:\\
	((... \textcolor{blue}{Each nuclear spin is identified with two hyperfine parameters $A_{zz}$ and $A_{zx}$, which shows coupling to the central electron spin. Hence, characterizing this 23-nuclear spins requires identifying 56 hyperfine coupling parameters.} ...)) \\
	
	((...\textcolor{blue}{which is the inner product of precession axes of a nuclear spin condition on the electron spin sublevel. This expression indicates the advantage of using DD sequence for sublevels where either $s_0$ or $s_1$ is zero, only first order parallel hyperfine coupling $\frac{A_{zz}}{\omega_L}$ play a role. On the other hand, in spin-1/2 systems ($s_0=-\frac{1}{2}$ and $s_1=\frac{1}{2}$), the first order perturbation correction with respect to hyperfine coupling vanishes and only second order perturbation term with respect to parallel $(\frac{A_{zz}}{\omega_L})^2$  and perpendicular coupling $(\frac{A_{zx}}{\omega_L})^2$ remain present. This indicates weak sensitivity of DD sequence on different nuclear spins}...))\\
	
	Also the following text is added to the caption of Fig 2:\\
	((... \textcolor{blue}{It indicates the covariance between two parameters. The main diagonal appears as each parameter is correlated with itself. The side-diagonals indicate the two hyperfine parameters that belong to the same nuclei.} ...)) \\
	
	We remark that no unit for Fig. 2c,d and Fig 3c,d are needed since these figures visualize the Fisher Information matrix elements. 
	
	2. \\
	
	3. We thank the referee for noting this point. We fixed this issue in Eq. 4 and Eq. 7  \\
	
	4. \textcolor{red}{We need to either add a reference or a small paragrah saying why Secular is applied and a $U_{int}$ to rotation frame, now it is too rude answer}The Hamiltonian in Eq. 1 is very well-understood in this community [ref here]. The Hamiltonian conventionally is written in the rotating frame of applied microwave and consist of Electron and nuclear spin Zeeman energy and dipole interaction between them. As it is mentioned in the text, the secular approximation is considered so terms like $S_x, S_y$ do not appear. Traditionally, the nuclear spins' frame are rotated such that $I_y$ term do not appear. This Hamiltonian applied to any two-level system, e.g., this could be any two spin sublevels in Silicon Carbide (S=3/2)\\
	
	5. We thank the referee for pointing out labels for subfigures (pdf is corrected but latex does not show). In the revised version of the manuscript we fixed this issue. 
	In Fig. 3, we are discussing 5pESEEM sequence mostly for Electron spin 1/2 system. 
	We did not include FFT of 5pESEEM sequence for spin 1 system because it works just as good as DD sequence, and does not bring any additional information. 
	However, we believe, that the Fisher information obtained for spin 1 system in Fig. 3c is good metrics to compare with the result of electron spin 1/2 system in Fig 3d.  \\
	
	6. \textcolor{red}{Initial state should be initialised into superposition of two eigenstates, s0,s1, which can be any, right?}The initial state for the Ramsey sequence can be any of the two sub-levels of spin system. Indeed, the grey exponential decay is not sufficient to describe spin relaxations as mentioned in the text, but it shows the time scale of the decay. We thank the referee for this point and we decided to remove this exponential decay to avoid confusion. We have added this correction to the main text.
	\textcolor{blue}{((..initial state is XXX..))}.\\
	 
	7. We thank the referee for pointing out this issue. We added more explanation to the last paragraph in the Appendix B.
	
	((...\textcolor{blue}{The blind spot term depends on both $\tau_1$ and $\tau_2$. It means if one resonant frequency is blinded, the other resonant frequency and all the multiple quantum resonances also vanishes. This can be used as a manifestation that which two peaks in the frequency spectrum are originating from the same nuclei. In other words, choosing $\tau$ such that a particular resonant frequency in the spectrum is blinded, the other resonant frequency of the same nuclear spin will also be blinded and both of them disappear from the frequency spectrum together. Moreover, this is very helpful especially in the presence of strongly coupled nuclear spin that suppress other nuclei. By sweeping $\tau_1$ and $\tau_2$, one can go through different bright and blind spots of each nuclear spin, and observe correlations between different peaks.}...))\\
	
	8. This equations are valid as long as the condition stated before Eq. 1 are valid. Most importantly secular approximation is crucial, and also the nuclear spin should be spin 1/2 like Carbons. The spin of electron spin is not playing an important role since various allowed microwave transitions can be well separated and pulses resonant with only an effective two level system can be considered.\\
	
	9. In this work, we address nuclear spins only with microwave driving of the electron spin (\textcolor{red}{did we mention this in the text?}). Applying another radio frequency (RF) driving adds another experimental challenge to the problem and goes beyond the scope of this work. In this case, RF spectroscopy to find and drive individual nuclear spins is required, which can add the required time to image nuclear spins, which was successfully demonstrated in various works for spin 1, 3/2,  \cite{DDRF, ENDOR, SSRV2}. We believe that it will as well work well for spin $S=1/2$, as it probes the transition frequency of the nuclear spin in $m_s=1/2$ or $m_s=-1/2$ subdomains, making it sensitive to the first order of the $A_{zz}$ coupling. Comparison of various RF assisted protocols goes beyond the scope of the current work and requires further investigation.\\

	\color{red} \section{Report of the Second Referee}
	The authors propose a blueprint for efficient nuclear spin
	characterization with color centers. The authors claim that they
	propose a more straightforward approach for determining the hyperfine
	interactions among each nuclear and the electron spin. The manuscript
	is written clearly by comparing dynamical decoupling (DD) and Electron
	Spin Echo Envelop Modulation (ESEEM), the numerical results are valid
	and may could be of use in other future experiments. I think the
	results obtained by the authors are of interest for Phys. Rev. B
	readers, so I support the publication in its present form.
	
	As some minor remarks which the authors may take into account to
	beatify their study further, I would like to mention:
	
	1) There are no a, b, c and d in Fig. 2, Fig. 3 and Fig.5. There is
	lack of definitions and units of “Hyperfine Parameters” in abscissa in
	Fig. 2 and Fig. 3. This sentence is confusing: “The first 23 hyperfine
	parameters are Azz and the second 23 parameters are Azx of the nuclear
	spin register.” What are the specific meanings of hyperfine parameters
	along horizontal axis and vertical axis in Figs2?
	
	2) Are there any reasons for the exponential decay part in gray for
	almost all the equations? It is a little bit strange, and the
	exponential decay part looks not in same size as well.
	
	3) Styles of references [9], [14], [19] and [21] are not uniform with
	other references.
	
	\color{black} \section*{Reply to the Second Referee}
	We appreciate the comments made by the referee. We clarify each point as follows:\\
	1. We thank the referee for pointing out labels for subfigures (pdf is corrected but latex does not show). We fixed this issue. Fig. 5 does not require subfigure labelling. We provided more explanation regarding the hyperfine coupling parameters in the text:
	((... \textcolor{blue}{the summation is over $\tau$, the pulse timing used in the sequence, and $A$ is the array including all hyperfine coupling parameters. For the considered nuclear spin register, $A$ includes 56 hyperfine coupling parameters, the first 23 parameters are $A_{zz}$ of different nuclear spins and the second 23 parameters are $A_{zx}$ respectively, resulting in the Fisher information as a 56 by 56 Matrix.} ...)) \\   
	
	2. Indeed, the grey exponential decay is not sufficient to describe spin relaxations as mentioned in the text, but it shows the time scale of the decay. We thank the referee for this point and we decided to remove this exponential decay to avoid confusion.\\
	
	3. We thank the referee for noting this and fixed the references in the revised version of the manuscript. 
	

	\color{red} \section{Report of the Third Referee}
	Zahedian et al. present a theoretical study on a the appropriateness of
	various methods of nuclear spin spectroscopy using electron spins in
	solid state materials as probes. They compare dynamical decoupling
	based techniques and consider how using a spin-1/2 system as a probe
	differs to the more common case of a spin-1 (such as the NV center in
	diamond). As the authors note, spin-1/2 electron spin systems are
	emerging in materials such as diamond and silicon and so expanding the
	existing nuclear spectroscopy machinery to this case will be of
	interest to the community.
	
	The paper and its main result that correlation-style dynamical
	decoupling sequences are most appropriate for performing nuclear spin
	spectroscopy with spin-1/2 electron spin systems should be of interest
	to the community, and so in principle I think it is appropriate to
	publish in PRB. In parts I found the arguments a little unclear and so
	I would recommend some minor revisions before publication. Some
	specific points:
	
	1) It is not clear to me why the correlation sequence performs much
	worse for the spin-1 probe (comparing Fig 2(c) and 3(c)) and there is
	not much discussion of this point in the text. This result also seems
	to conflict with Fig 5. It also does not seem like the spin-1 probe
	should be worse than spin-1/2 in this case. Also, what are the
	parameters used to generate the plots Fig 3c,d - is it simulating the
	72 pulse DDESEEM?
	
	2) It would be interesting to know, roughly, how the results of e.g.
	Fig 5 scale with properties such as probe T1 and T2. Does this
	strongly impact the optimal sequence choice? Some spin-1/2 systems may
	have short T1 times as they are able to cross relax with other g=2
	spins present in the crystal (or on the diamond surface, for
	instance).
	
	3) One goal of mapping e.g. a nuclear spin cluster is to also use the
	central electron spin to selectively manipulate individual nuclear
	spins to use as ancillae or implement quantum information protocols.
	Is it straightforward that the spectral resolution offered by the
	correlation style measurements also allow this sort of control? At
	minimum the extra time overhead from using the longer sequence would
	need to be considered in practice. Could the authors comment on the
	outlook here?
	
	4) Minor comment on figures: some labels are missing and fig 4 appears
	a bit blurry.
	
	\color{black} \section*{ Reply to the Third Referee}
	We appreciate the comments made by the referee. We clarify each point as follows:\\
	
	1. Characterization of spin-1 system with both DD and DDESEEM works similarly well. Indeed, it is important to observe the diagonal and side-diagonals are distinguishable from in the Fisher Information matrix, which in both Fig. 2c and 3c are clear. However, the color scale is levelled such that in each sequence, the perfomance of spin 1/2 can be compared with spin 1. Fig. 3c, 3d are Fisher Information matrix simulated for 5pESEEM. For clarity we added this point in the caption.\\
	
	2. $T_1$ limits the resolution of the frequency spectrum. However, $T_2$ limits the weakest nuclear spin that can be addressed. Ideally, by increasing number of $\pi$ pulses, this limit can be pushed toward $T_1$. (did we mention this in the text?) While short transverse relaxation time tolerable, having short longitudinal relaxation time basically means that nuclear spin is not a desired quantum memory. \\
	
	3. Using correlation sequences as quantum gates can be challenging as they are not unitary gates due to transverse relaxation of the spins.\\
	
	4. We thank the referee for pointing out labels for subfigures (pdf is corrected but latex does not show). We fixed this issue. Fig. 5 does not require subfigure labelling.


\end{document}