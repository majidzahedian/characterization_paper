
\documentclass[12pt]{amsart}
\usepackage{geometry} % see geometry.pdf on how to lay out the page. There's lots.
\geometry{a4paper} % or letter or a5paper or ... etc
% \geometry{landscape} % rotated page geometry

% See the ``Article customise'' template for come common customisations
% delete this line to display the current date

%%% BEGIN DOCUMENT
\begin{document}

Dear editor,

Accurate characterisation of the nuclear spin environment is pivotal for understanding of an environment of a solid state register. 
This information is crucial for purificatino of the register operation but also is a resource for extending the regsiter itself, utilising the spins in the environment as so. 
As recently the sucess of the pushing the registers using environmental nuclear spins reached around 50 qubits with most studied NV centers in diamond, a surve for spin characterisation techniques suitable for many other colour centers is long awaited in our opinion. 
Partially this is also motivated by the challenge and interest which a simplest S=1/2 electron spin systems present. 
The most succesfull and widespread technique - namely the dynamical decoupling is not applicable in their case. 
Quite interestingly, a detour to the world of convential EPR type techniques, such as ESEEM, (5p, 3p and more exotic, such as HYSCORE) appears usefull. 
Now applied at the level of single S=1/2 systems, it might overperform the conventional DD sequences in terms of resolution and sensitivity to the hyperfine parameters of the nuclear spins in the bath.  
We quantitatively demonstrate how such sequence performs in various scenarious on the example of the well studied 27 nuclear spin bath.
We utilise the Cramer Rao bound for estimating the uncertainty and number of identified nuclear spins using various techniques. 

We strongly believe that our work is of importance for future development of colour centers with S=1/2 electron spins, such as Group IV heavy elements split vacancies, as well as newly developed T centers in Silicon and many others. 
\\
\\
On behalf of all authors, \\
Vadim Vorobyov
\end{document}